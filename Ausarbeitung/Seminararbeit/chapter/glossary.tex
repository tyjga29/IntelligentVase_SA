
\newacronym{zahl}{ZAHL}{Zahlen}

\newglossaryentry{RVID}
{
	type=main,
    name=RetailVariantId,
    description={Id auf größen Ebene},
    plural={RetailVariantIds}
}

\newglossaryentry{collec}
{
	type=main,
    name=Collection,
    description={Speichermedium für MongoDB Datenbanken}
    plural={Collections}
}

\newglossaryentry{Fuzzy Controller}{
    type=main,
    name=Collection,
    description={Ein Fuzzy-Controller kategorisiert mehrere analoge Werte um eine gewünschte Aktion auszuführen. Beispielsweise erhält ein Klärwerk die Input-Werte Verschmutzung des Wassers und Bedarf. Bei viel Verschmutzung wenig Bedarf wird nur eine geringe Menge in das Klärwerk gelassen. Bei einem hohen Bedarf und wenig Verschmutzung eine große Menge.}
    plural={Collections}
}

\newglossaryentry{Machine Learning}
{
	type=main,
    name=Machine Learning,
    description={Wissenschaft, welche sich mit der Erstellung von stochastischen Algorithmen befasst, welche ohne explizite Instruktionen Daten auswertet}
}

\newglossaryentry{Gamification}
{
	type=main,
    name=Gamification,
    description={Eine Erfahrung mit verschiedenen Techniken von Spielen auszubauen, um Nutzuende zu motivieren}
}

\newglossaryentry{Greedy Algorithmus}
{
	type=main,
    name=Greedy Algorithmus,
    description={Einfache Art von problemlösenden Algorithmen, sie suchen nicht die beste Lösung, sondern eine aktzeptable in einem annehmbaren Zeitraum}
}

\newglossaryentry{Entscheidungsbaum}
{
	type=main,
    name=Entscheidungsbaum,
    description={Hierarchische Baumstruktur, um Entscheidungen zu fällen}
}

\newglossaryentry{ANN}
{
	type=main,
    name=Künstliche Neuronale Netzwerke,
    description={Machine Learning Algorithmen, welche sich auf die Funktionsweise des menschlinen Gehirns berufen}
}

\newglossaryentry{Fuzzy Logic}
{
	type=main,
    name=Fuzzy Logic,
    description={Alternative zu Boolean Entscheidungen, indem Entscheidungsspannen festgelegt werden. Welche Temperatur kann als warm oder kalt anerkannt werden}
}

\newglossaryentry{Trainingsset}
{
	type=main,
    name=Trainingsset,
    description={Beispiele für ein Machine-Learning-Algorithmus, von dem er Regeln ableiten kann}
}

\newglossaryentry{MorphA}
{
	type=main,
    name=Morphologische Analyse,
    description={Aus der Produktentwicklung stammendes System, bei dem mehrere Herangehensweisen benutzt werden, um alle Seiten eines Konzeptes vorurteilslos zu betrachten}
}

\newglossaryentry{SelecSea}
{
	type=main,
    name=Selective Search,
    description={Wenig aufwendige Methode zur Objekterkennung, basierend auf der Forschung von Felzenszwalb und Huttenlocher}
}