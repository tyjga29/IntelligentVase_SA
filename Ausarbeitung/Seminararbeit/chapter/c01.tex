\section{Einleitung}\label{sec_1}

Intelligente Maschinen übernehmen schon lange die Pflanzenpflege für landwirtschaftliche Unternehmen. Dies soll Kosten senken, Ressourcen schonen, den Ablauf produktiver gestalten und die Qualität der Ernte erhöhen. Bereits jetzt butzen über 20 Prozent aller landwirtschaftlichen Beriebe Farm-Management-Software, rund fünf Prozent setzen auf vollautomatisierte Technologien.\cite{automate2023impactautomationagriculture} Die Technologien sind hierbei weitreichend: Roboter, automatisierte Gewächshäuser und Bewässerungsanlagen und Datenauswertungen tragen dazu bei, gesunde Pflanzen effizient zu kultivieren. 

Derartige Ansätze sind auch im privaten Gebrauch sinnvoll. Eine Studie aus den USA schätzt den Marktwert der amerikanischen Hauspflanzindustrie auf 16 Milliarden Dollar ein\cite{narishkin2022houseplants}. Die Pflege von Hauspflanzen ist jedoch anspruchsvoll; in den Niederlanden sterben jedes Jahr 16 Millionen Hauspflanzen, hauptsächlich aufgrund mangelnden Wissens über die Bedürfnisse verschiedener Pflanzen.\cite{hollandtimes2024plants} Diese Werte sind repräsentativ für den Großteil der westlichen Länder. Automatisierte Pflanzenpflege könnte helfen, Pflanzen auch ohne spezielles Fachwissen gesund zu halten. Zudem kann sie Menschen Arbeit abnehmen, da die Pflege zeitaufwendig ist und aktive Beteiligung erfordert. Es ist notwendig, Wissen über Pflanzenpflege zu erwerben, die Pflanzen zu beobachten, ihre Reaktionen auf Umwelteinflüsse zu verstehen, Krankheiten zu identifizieren und regelmäßig Tätigkeiten wie Düngen und Gießen durchzuführen. Durch die teilweise Automatisierung dieser Aufgaben bleibt den Nutzenden mehr Zeit für andere Aktivitäten.

Automatisierung und Unterstützung im privaten Bereich könnten somit nicht nur den Aufwand für weniger engagierte Pflanzenbesitzende reduzieren, sondern auch die Verschwendung gesunder Pflanzen verringern, die sonst neu gekauft werden müssten..

\subsection{Abgrenzung zu anderen Projekten und Zielstellung}
Dieses Projekt unterscheidet sich von anderen Produkten durch seine vollständige Sprachsteuerung und die Berücksichtigung verschiedener Nutzergruppen. Es soll sowohl für Enthusiasten relevante Informationen und Zusammenhänge zwischen Umweltparametern aufdecken als auch die Pflanzenpflege nahezu vollständig übernehmen. Es wird lediglich ein Sprachassistent benötigt, visuelle Eingaben sind nicht erforderlich. Das Gerät soll den Nutzenden Zeit ersparen und keine zusätzliche Aufmerksamkeit erfordern. Interessierten bietet es wertvolle Informationen, ohne sie in redundanten Informationsflüssen zu verstricken.

Durch den Einsatz verschiedener KI-Techniken soll die Maschine optimale Ergebnisse erzielen und die Pflege an die Bedürfnisse jeder speziellen Pflanze anpassen. Kleine Veränderungen der Umweltparameter werden genutzt, um das Wachstum der Pflanzen zu optimieren. Nutzende sollen nur bei Interesse in diesen Prozess eingreifen.

Um diesen Ansprüchen gerecht zu werden, reicht ein lokales System, das isoliert von der Außenwelt agiert, nicht aus. Berechnungen und Erfahrungswerte von vielen verschiedenen Geräten sollen zusammengetragen und gespeichert werden. So können Optimalwerte für neue Pflanzengattungen erstellt und jederzeit angepasst werden. Werte wie die Wasserausgabe oder die Zeitabstände zwischen Berechnungen können flexibel verändert werden, um das System zu verbessern. Die gesammelten Informationen können auch zur Erstellung weiterer Projekte und Studien genutzt werden.

\subsection{Aufbau der Arbeit}
In diesem Projekt wird ein Prototyp für eine intelligente Vase erstellt.

Zunächst werden andere Produkte und Projekte gesichtet, um die Anforderungen zu definieren und aus bereits gemachten Erfahrungen zu lernen. Mithilfe eines Mikrocontrollers und Sensoren werden verschiedene Umweltparameter gesammelt und in einer zentralen Datenbank gespeichert. Aus diesen Daten werden Berechnungen durchgeführt, um sie mit optimalen Wachstumswerten zu vergleichen und sinnvolle Anweisungen zu generieren. An den Mikrocontroller wird eine Pumpe angeschlossen, die die Pflanze automatisch bewässert.

Anschließend wird das Projekt hinsichtlich der Anforderungen, Codequalität und Funktionsweise evaluiert.

Abschließend werden die Ergebnisse zusammengefasst und ein Ausblick auf das weitere Vorgehen gegeben.